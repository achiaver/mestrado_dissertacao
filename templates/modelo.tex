% ---------------------------------------------------------------
% ---------------------------------------------------------------
% Check out config/disclaimer.tex
% ---------------------------------------------------------------
% ---------------------------------------------------------------
%
\documentclass[a4paper,12pt,oneside,onecolumn,final,fleqn]{repUERJ}
% ---
% Pacotes fundamentais 
% ---
\usepackage[brazil]{babel}  % adequação para o português Brasil
\usepackage[utf8]{inputenc} % Determina a codificação utilizada
                            % (conversão automática dos acentos)
\usepackage{makeidx}        % Cria o índice
\usepackage{hyperref}       % Controla a formação do índice
\usepackage{indentfirst}    % Indenta o primeiro paragrafo de
                            % cada seção.
\usepackage{graphicx}       % Inclusão de gráficos
\usepackage{subfig}
\usepackage{multirow}
\usepackage{amsmath,amssymb}  % pacotes matemáticos
\usepackage[alf]{abntex2cite} % pacote para citacoes autor-data
% \usepackage[num]{abntex2cite} % pacote para citacoes com numeracao
% \citebrackets[]

\pdfstringdefDisableCommands{\let\uppercase\relax}
\pdfstringdefDisableCommands{\let\uline\relax}

\usepackage[font=default,frame=no]{repUERJformat} % pacote para as 
                                                  % normas da UERJ
% ---
% pacote auxiliar para elaboração de pseudocódigos
% este pacote pode ser retirado caso nao se planeje
% elaborar pseudocódigos
% ---
\usepackage[dots=yes]{repUERJpseudocode}

\usepackage[maxfloats=25]{morefloats}
\usepackage{array}
\setlength\extrarowheight{2pt}

% ----------------------------------------------------------------
% Este trecho de comandos pode ser apagado
% ----------------------------------------------------------------
\usepackage[brazilian,hyperpageref]{backref}
% Configurações do pacote backref
% Usado sem a opção hyperpageref de backref
\renewcommand{\backrefpagesname}{Citado na(s) página(s):~}
% Texto padrão antes do número das páginas
\renewcommand{\backref}{}
% Define os textos da citação
\renewcommand*{\backrefalt}[4]{
\ifcase #1 %
Nenhuma citação no texto.%
\or
Citado na página #2.%
\else
Citado #1 vezes nas páginas #2.%
\fi}%
% ----------------------------------------------------------------


% ***************************************************************
% Informações de autoria e institucionais
% ***************************************************************

%----------------------------------------------------------------
% Imagens pretextuais (precisam estar no mesmo diretório deste arquivo .tex)
%----------------------------------------------------------------

\logo{logo_uerj_cinza.png}
\marcadagua{marcadagua_uerj_cinza.png}{1}{160}{255}

%----------------------------------------------------------------
% Informações da instituição
%----------------------------------------------------------------

\instituicao{Universidade do Estado do Rio de Janeiro}
            {Centro de Tecnologia e Ciências} 
            {Instituto de Matemática e Estatística} 
            %{Patrono} 

%----------------------------------------------------------------
% Informações da autoria do documento
%----------------------------------------------------------------

% A seguir, devem ser atualizadas as informações particulares do  trabalho, autor(a) e orientação; assim como escolher se está se elaborando uma tese de doutorado ou uma dissertação de mestrado.

\autor{Nome}{Sobrenome}
      {Iniciais do nome} % iniciais do nome

\titulo{Título do trabalho}
\title{Título do trabalho em inglês}

% se não for usar a quarta palavra chave, deixar o campo vazio: {}
\palavraschaves{Primeira palavra-chave}
               {Segunda palavra-chave}
               {Terceira palavra-chave}
               {Quarta palavra-chave (opcional) ou vazio}

\keywords{First keyword}
         {Second keyword}
         {Third keyword}
         {Fourth keyword or empty}

\orientador{Cargo Titulação} 
           {Nome}{Sobrenome} 
           {Unidade -- UERJ} 

\coorientador{Cargo Titulação} 
             {Nome}{Sobrenome} 
             {Unidade -- Instituição} 

%----------------------------------------------------------------
% Grau pretendido (Doutor, Mestre, Bacharel, Licenciado) e Curso
%----------------------------------------------------------------

% Para dissertação de mestrado utilizar o \grau{Mestre} e comentar  o \grau{Doutor}.
\grau{Mestre} 

% Para tese de doutorado utilizar o \grau{Doutor} e comentar o \grau{Mestre}.
% \grau{Doutor}

\curso{Ciências Computacionais}


%----------------------------------------------------------------
% Informações adicionais (local, data e paginas)
%----------------------------------------------------------------

\local{Rio de Janeiro} 
\data{dd}{mês}{aaaa} 

% ***************************************************************
% Configurações de aparência do PDF final
% ***************************************************************

% alterando o aspecto da cor azul
%\definecolor{blue}{RGB}{41,5,195}
%\definecolor{apricot}{RGB}{251,206,177}

% informações do PDF
\hypersetup{
  unicode=false,
  pdftitle={\UERJtitulo},
  pdfauthor={\UERJautor},
  pdfsubject={\UERJpreambulo},
  pdfkeywords={PALAVRAS-CHAVES:}{ \chaveA}{ \chaveB}{ \chaveC}{ \chaveD},
  pdfproducer={\packagename}, % producer of the document
  colorlinks=true,   % false: boxed links; true: colored links
  linkcolor=black,   % color of internal links blue
  citecolor=black,   % color of links to bibliography blue
  filecolor=black,   % color of file links magenta
  urlcolor=black,
  bookmarksdepth=4,
%   backref=true,
%   pagebackref=true,
%   bookmarks=true,
}

% ***************************************************************
% Índice remissivo
% ***************************************************************

% \makeindex % compila o índice; se não for usar, comentar

% ***************************************************************
% Fim do preâmbulo
% ***************************************************************

% ----------------------------------------------------------
\begin{document}
\citeoption{abnt-options4}
% ----------------------------------------------------------

% ----------------------------------------------------------
% ELEMENTOS PRE-TEXTUAIS
% ----------------------------------------------------------

\frontmatter % inicia a área dos elementos pré-textuais

% O tipo de documento (tese ou dissertação) deve ser definido após a linha 165

% ----------------------------------------------------------
% Capa e a folha de rosto
% ----------------------------------------------------------

\capa
\folhaderosto

% ----------------------------------------------------------
% Inserir a ficha catalográfica
% ----------------------------------------------------------

% A biblioteca deverá providenciar a ficha catalográfica. 
% Salve a ficha no formato PDF. Use o nome do arquivo PDF 
% como argumento do comando. 
% Exemplo: ficha catalográfica no arquivo 'ficha.pdf'
%     \fichacatalografica{ficha.pdf}
%
% Enquanto não possuir a ficha catalográfica, use o comando sem
% argumentos.
% ---

\fichacatalografica{}


% ----------------------------------------------------------
% Folha de aprovação
% ----------------------------------------------------------

\begin{folhadeaprovacao}
  \assinatura{Cargo Título Nome Completo}
             {Unidade -- Instituição}
  \assinatura{Cargo Título Nome Completo}
             {Unidade -- Instituição}
  \assinatura{Cargo Título Nome Completo}
             {Unidade -- Instituição}
  \assinatura{Cargo Título Nome Completo}
             {Unidade -- Instituição}
\end{folhadeaprovacao}


% ----------------------------------------------------------
% Dedicatória
% ----------------------------------------------------------

\pretextualchapter{Dedicatória}
\vfill
Texto da dedicatória (opcional).


% ----------------------------------------------------------
% Agradecimentos
% ----------------------------------------------------------

\pretextualchapter{Agradecimentos}

Texto de agradecimento (opcional).


% ----------------------------------------------------------
% Epigrafe (opcional)
% ----------------------------------------------------------

\pretextualchapter{}
  \vfill
  \begin{flushright}
 Pensamento, reflexão (opcional). \\  %citação sem aspas   
    \textit{autor}
  \end{flushright}


% ----------------------------------------------------------
% RESUMO
% ----------------------------------------------------------

\pretextualchapter{Resumo}
\referencia % linha em branco depois


Texto do resumo em português.


\imprimirchaves % linha em branco antes

% ----------------------------------------------------------
% Abstract
% ----------------------------------------------------------

\pretextualchapter{Abstract}
\reference % linha em branco depois


Texto do resumo em inglês.


\printkeys % linha em branco antes

% ----------------------------------------------------------
% Listas de ilustrações e tabelas
% ----------------------------------------------------------

\listadefiguras    %\begin{figure}{largura}...\end{figure}
\listadetabelas    %\begin{table}{largura}...\end{table}
% \listadegraficos   %\begin{graph}{largura}...\end{graph}

% ----------------------------------------------------------
% Outras listas
% ----------------------------------------------------------

\listadealgoritmos % opcional %\begin{algorithm}...\end{algorithm}

% ----------------------------------------------------------
% Lista de abreviaturas e siglas (opcional)
% ----------------------------------------------------------

\pretextualchapter{Lista de abreviaturas e siglas}
    \abreviatura{sigla 1}{por extenso}
    \abreviatura{sigla 2}{por extenso}
    \abreviatura{sigla 3}{por extenso}

% ----------------------------------------------------------
% Lista de simbolos (opcional)
% ----------------------------------------------------------

\pretextualchapter{Lista de símbolos}
    \simbolo{$simbolo 1$}{significado e/ou valor}
    \simbolo{$simbolo 2$}{significado e/ou valor}
    \simbolo{$simbolo 3$}{significado e/ou valor}

% ----------------------------------------------------------
% Sumario
% ----------------------------------------------------------

\sumario

% ----------------------------------------------------------
% ELEMENTOS TEXTUAIS
% ----------------------------------------------------------

\mainmatter % inicia a área de desenvolvimento do conteúdo

%================================================================
\chapter*{Introdução} % Não deve ser numerado.
%================================================================

No âmbito da comunidade acadêmica, a importância de estabelecer diretrizes, padrões e normas que constituem uma ação fundamental para garantir a localização e a recuperação do conhecimento produzido é reconhecida principalmente por ser a Universidade locus permanente de produção científica~\cite{bib:roteiro}.

As teses e dissertações da Universidade do Estado do Rio de Janeiro, destacam-se pela qualidade do conteúdo e pela estrutura da apresentação. Organizar essa estrutura e padronizar a apresentação são processos que dependem de atualização constante devido às novas Tecnologias de Informação e Comunicação (TICs), da produção de informação que utiliza esses recursos e, também, das normas da Associação Brasileira de Normas Técnicas (ABNT)~\cite{bib:roteiro}.

Este modelo tem por objetivo padronizar os trabalhos realizados no Programa de Pós-Graduação em Ciências Computacionais (PPG-CCOMP). O modelo utilizado foi criado por Dr. Luís Fernando de Oliveira, professor adjunto do Dep. de Física Aplicada e Termodinâmica da Universidade do Estado do Rio de Janeiro e, adaptado para os cursos de Mestrado e Doutorado do PPG-CCOMP por M.Sc. Michelle Lau.

Foram introduzidas algumas regras gerais para apresentação no próximo capítulo. Para mais detalhes sobre a apresentação, consultar o Roteiro para apresentação das teses e dissertações da Universidade do Estado do Rio de Janeiro através do link: \url{http://www.bdtd.uerj.br/roteiro_uerj_web.pdf}.


%================================================================
\chapter{Regras Gerais de Apresentação}
%================================================================

São as regras que definem como a parte gráfica do documento
deve ser apresentada, incluindo as ilustrações, as equações e as abreviaturas e siglas que porventura existam.


%----------------------------------------------------------------
\section{Abreviaturas e siglas}
%----------------------------------------------------------------

São utilizadas com o objetivo de se evitar a repetição de palavras ou de expressões que aparecem com frequência no texto. Quando as abreviaturas ou siglas forem citadas no texto pela primeira vez, devem aparecer entre parênteses após o seu significado por extenso.

\textbf{Exemplo:} A Universidade do Estado do Rio de Janeiro (UERJ) integra o consórcio da Biblioteca Digital Brasileira de Teses e Dissertações (BDTD). Assim, a UERJ faz parte da BDTD nacional, coordenada pelo Instituto Brasileiro de Informação em Ciência e Tecnologia (IBICT).


%----------------------------------------------------------------
\section{Equações e fórmulas}
%----------------------------------------------------------------

As equações e fórmulas devem aparecer destacadas no texto, alinhadas na margem esquerda, numeradas em algarismos arábicos, estes
alinhados na margem direita e entre parênteses. 

Alguns exemplos usando ambientes para inserir equações matemáticas:

\begin{align}
    & X_{t+1} = kx_t\left(1-x_t\right) \notag \\
    & E = mc^2\\
    & F = G \dfrac{m_1 m_2}{d^2}
\end{align}

\begin{gather}
    f(x)=\left\{\begin{aligned}
    & y_{ij} = 0\\
    & x = h + v
\end{aligned}\right.
\end{gather}

\begin{equation} 
    H = - \sum p(x) \log p(x)
\end{equation}

\begin{equation*}
    \nabla \times H = \dfrac{1}{c}\dfrac{\partial E}{\partial t},~~ \nabla\cdot H = 0
\end{equation*}

%----------------------------------------------------------------
\section{Ilustrações}
%----------------------------------------------------------------

Têm por objetivo exemplificar e/ou esclarecer o assunto que está
sendo abordado. Deve-se evitar o uso de ilustrações que não tenham sido objeto do trabalho.

\textbf{Exemplos:}

\begin{verbatim}
\begin{figure}[htb]{14cm}
\caption{Aqui entra o título da figura -- deve ser curto e objetivo}
\label{fig:fig1}
\fbox{\includegraphics[width=0.45\hsize]{logo_uerj_cinza.png}}%
\legend{Aqui entra a legenda da figura. Deve ser simples e sucinta
mas que garanta a compreensão da ilustração.
Qualquer explicação mais longa deve estar no corpo do texto.}
\source{Citação da fonte ou `O autor/A autora, ano'.}
\end{figure}
\end{verbatim}

O código acima produz a Figura~\ref{fig:fig1}.

\begin{figure}[htb]{14cm}
\caption{Aqui entra o título da figura -- deve ser curto e objetivo}\label{fig:fig1}
\fbox{\includegraphics[width=0.45\hsize]{config/UERJ/logo_uerj_cinza.png}}%
 \legend{Aqui entra a legenda da figura. Deve ser simples e sucinta mas que garanta a compreensão da ilustração.
Qualquer explicação mais longa deve estar no corpo do texto.}
\source{Citação da fonte ou `O autor/A autora, ano'.}
\end{figure}

Com o comando $\backslash$\texttt{subfloat[][]}\{\texttt{...}\} dentro do $\backslash$\texttt{begin}\{\texttt{figure}\}, é possível organizar mais de uma imagem no mesmo título. Cada ``subimagem'' será referenciada com uma letra a começar com \texttt{(a)}. Veja as figuras \ref{subrotulo1}, \ref{subrotulo2} e \ref{subrotulo3}.
\begin{verbatim}
\begin{figure}[htb]{16cm}
  \caption{Apresentações do logo da UERJ.} 
  \label{outro.rotulo}
  \subfloat[][]{\label{subrotulo1}%
    \fbox{\includegraphics[width=0.45\hsize]
            {logo_uerj_cinza.png}}}\hfill
  \subfloat[][]{\label{subrotulo2}%
    \fbox{\includegraphics[width=0.45\hsize]
            {marcadagua_uerj_cinza.png}}}\\
  \subfloat[][]{\label{subrotulo3}%
    \fbox{\includegraphics[width=0.45\hsize]
            {logo_uerj_cor.jpg}}}\hfill
  \legend{Logomarca da UERJ em diferentes 
    apresentações de cor: \subref{subrotulo1} 
    em tons de cinza (a partir da imagem colorida);
    \subref{subrotulo2} como marga d'água (a partir 
    da imagem preto e branco; \subref{subrotulo3} 
    colorido.}
  \source{\citeauthor{bib:logomarca}, 
          \citeyear{bib:logomarca}; 
          \citeauthor{bib:logomarca}, 
          \citeyear{bib:logomarca}.}
\end{figure}
\end{verbatim}

Não vale a pena brigar com o \LaTeX\ para posicionar a figura onde você quer. Normalmente, o \LaTeX\ colocará a figura na página seguinte, salvo exceções. Isto tem  a ver com a proporção de texto e figura presentes na página. Quando a figura ocupa uma área da página muito superior ao correspondente de texto, o \LaTeX\ deixa a figura isolada em uma nova página como, por exemplo, a Figura~\ref{fig:outro.rotulo}.

\begin{figure}[htb]{16cm}
  \caption{Apresentações do logo da UERJ.} 
  \label{fig:outro.rotulo}
  \subfloat[][]{\label{subrotulo1}%
    \fbox{\includegraphics[width=0.45\hsize]
            {cofig/UERJ/logo_uerj_cinza.png}}}\hfill
  \subfloat[][]{\label{subrotulo2}%
    \fbox{\includegraphics[width=0.45\hsize]
            {config/UERJ/marcadagua_uerj_cinza.png}}}\\
  \subfloat[][]{\label{subrotulo3}%
    \fbox{\includegraphics[width=0.45\hsize]
            {config/UERJ/logo_uerj_cor.jpg}}}\hfill
  \legend{Logomarca da UERJ em diferentes 
    apresentações de cor: \subref{subrotulo1} 
    em tons de cinza (a partir da imagem colorida);
    \subref{subrotulo2} como marga d'água (a partir 
    da imagem preto e branco; \subref{subrotulo3} 
    colorido.}
  \source{\citeauthor{bib:logomarca}, 
          \citeyear{bib:logomarca}; 
          \citeauthor{bib:logomarca}, 
          \citeyear{bib:logomarca}.}
\end{figure}


%----------------------------------------------------------------
\section{Tabelas}
%----------------------------------------------------------------

Apresentam apenas informações estatísticas. As tabelas seguem as mesmas regras das ilustrações quanto a identificação, apresentação gráfica, localização e disposição.

\begin{table}[htb]{16cm}
    \caption{\protect Algumas disciplinas do Programa de Pós-Graduação em Ciências Computacionais.}
    \label{tab:casos_sarampo}
    \hfill\begin{tabular}{l|c}
        Disciplina & Grupo \\
        \hline 
         Álgebra Linear: Aspectos Teóricos e Computacionais & I\\
         Algoritmos & I\\
         Programação Paralela e Distribuída & II\\
         Probabilidade e Estatística & II\\
         Redes Neuronais & III\\
         Visão Computacional & III\\
         Estudo Dirigido I & IV
    \end{tabular}\hfill
    \legend{Algumas das disciplinas ofertadas pelo Programa de Pós-Graduação em Ciências Computacionais e o grupo ao qual pertencem.} 
    \source{O autor, 2021.}
\end{table}


%----------------------------------------------------------------
\section{Referências}
%----------------------------------------------------------------

No arquivo \textit{exemplos.bib} podem-se observar alguns exemplos de como formatar as referências a serem utilizadas no trabalho. 

Para obter mais detalhes, acessar o Roteiro para apresentação das teses e dissertações da Universidade do Estado do Rio de Janeiro através do link: \url{http://www.bdtd.uerj.br/roteiro_uerj_web.pdf}.

%================================================================
\chapter*{Conclusão} %não deve ser numerado
%================================================================

Texto da conclusão.




% ----------------------------------------------------------------
% ELEMENTOS POS-TEXTUAIS
% ----------------------------------------------------------------

\backmatter % inicia a área dos elementos pós-textuais


%===========================================================
% Referencias via BibTeX
%===========================================================

\citeoption{abnt-options4}
\bibliography{abnt-options4,bibliografia}
% \bibliographystyle{abntex2-num}
% \bibliography{bibliografia}

%===========================================================
\postextualchapter*{Glossário} % elemento opcional
%===========================================================

\definicao{termo 1}{significado}
\definicao{termo 2}{significado}
\definicao{termo 3}{significado}

% ----------------------------------------------------------------
% Apêndices (opcionais)
% ----------------------------------------------------------------

\appendix % inicia os apêndices

%===========================================================
\postextualchapter{Primeiro apêndice}
%===========================================================

Os apêndices são elementos opcionais, complementares, de caráter informativo, elaborados pelo próprio autor, como, por exemplo, questionários, formulários, textos etc. Sua exclusão não prejudica o conteúdo do trabalho. Caso haja somente um apêndice, este não deve ser numerado. Para isso, é necessário comentar a linha $565$ do arquivo repUERJ.cls e retirar o comentário da linha $566$ do mesmo arquivo.

% ----------------------------------------------------------
\section{Primeira seção}
% ----------------------------------------------------------

Texto da primeira seção.

% ----------------------------------------------------------
\subsection{Primeira subseção}
% ----------------------------------------------------------

Texto da primeira subseção.

% ----------------------------------------------------------
\subsubsection{Primeira subsubseção}
% ----------------------------------------------------------

Texto da primeira subsubseção.


%===========================================================
\postextualchapter{Segundo apêndice}
%===========================================================

% ----------------------------------------------------------
\section{Primeira seção}
% ----------------------------------------------------------

Texto da primeira seção.

% ----------------------------------------------------------
\subsection{Primeira subseção}
% ----------------------------------------------------------

Texto da primeira subseção.

% ----------------------------------------------------------
\subsubsection{Primeira subsubseção}
% ----------------------------------------------------------

Texto da primeira subsubseção.

% ----------------------------------------------------------------
% Anexos (opcionais)
% ----------------------------------------------------------------

\annex % inicia os anexos

%===========================================================
\postextualchapter{Primeiro anexo}
%===========================================================

Os anexos são elementos opcionais, complementares, de caráter ilustrativo e/ou comprobatório do texto. O anexo difere do apêndice por não ser elaborado pelo autor da tese ou dissertação. Caso haja somente um anexo, este não deve ser numerado, como ocorre no apêndice.
Caso haja mais de um anexo, é necessário comentar a linha $575$ do arquivo repUERJ.cls e retirar o comentário da linha $574$ do mesmo arquivo.

% ----------------------------------------------------------
\section{Primeira seção}
% ----------------------------------------------------------

Texto da primeira seção.

% ----------------------------------------------------------
\subsection{Primeira subseção}
% ----------------------------------------------------------

Texto da primeira subseção.

% ----------------------------------------------------------
\subsubsection{Primeira subsubseção}
% ----------------------------------------------------------

Texto da primeira subsubseção.



% %===========================================================
% \postextualchapter{Segundo anexo}
% %===========================================================

% % ----------------------------------------------------------
% \section{Primeira seção}
% % ----------------------------------------------------------

% Texto da primeira seção.

% % ----------------------------------------------------------
% \subsection{Primeira subseção}
% % ----------------------------------------------------------

% Texto da primeira subseção.

% % ----------------------------------------------------------
% \subsubsection{Primeira subsubseção}
% % ----------------------------------------------------------

% Texto da primeira subsubseção.


\end{document}
