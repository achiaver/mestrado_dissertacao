



%================================================================
\chapter*{Introdução} % Não deve ser numerado.
%================================================================

No âmbito da comunidade acadêmica, a importância de estabelecer diretrizes, padrões e normas que constituem uma ação fundamental para garantir a localização e a recuperação do conhecimento produzido é reconhecida principalmente por ser a Universidade locus permanente de produção científica~\cite{bib:roteiro}.

As teses e dissertações da Universidade do Estado do Rio de Janeiro, destacam-se pela qualidade do conteúdo e pela estrutura da apresentação. Organizar essa estrutura e padronizar a apresentação são processos que dependem de atualização constante devido às novas Tecnologias de Informação e Comunicação (TICs), da produção de informação que utiliza esses recursos e, também, das normas da Associação Brasileira de Normas Técnicas (ABNT)~\cite{bib:roteiro}.

Este modelo tem por objetivo padronizar os trabalhos realizados no Programa de Pós-Graduação em Ciências Computacionais (PPG-CCOMP). O modelo utilizado foi criado por Dr. Luís Fernando de Oliveira, professor adjunto do Dep. de Física Aplicada e Termodinâmica da Universidade do Estado do Rio de Janeiro e, adaptado para os cursos de Mestrado e Doutorado do PPG-CCOMP por M.Sc. Michelle Lau.

Foram introduzidas algumas regras gerais para apresentação no próximo capítulo. Para mais detalhes sobre a apresentação, consultar o Roteiro para apresentação das teses e dissertações da Universidade do Estado do Rio de Janeiro através do link: \url{http://www.bdtd.uerj.br/roteiro_uerj_web.pdf}.


%================================================================
\chapter{Regras Gerais de Apresentação}
%================================================================

São as regras que definem como a parte gráfica do documento
deve ser apresentada, incluindo as ilustrações, as equações e as abreviaturas e siglas que porventura existam.


%----------------------------------------------------------------
\section{Abreviaturas e siglas}
%----------------------------------------------------------------

São utilizadas com o objetivo de se evitar a repetição de palavras ou de expressões que aparecem com frequência no texto. Quando as abreviaturas ou siglas forem citadas no texto pela primeira vez, devem aparecer entre parênteses após o seu significado por extenso.

\textbf{Exemplo:} A Universidade do Estado do Rio de Janeiro (UERJ) integra o consórcio da Biblioteca Digital Brasileira de Teses e Dissertações (BDTD). Assim, a UERJ faz parte da BDTD nacional, coordenada pelo Instituto Brasileiro de Informação em Ciência e Tecnologia (IBICT).


%----------------------------------------------------------------
\section{Equações e fórmulas}
%----------------------------------------------------------------

As equações e fórmulas devem aparecer destacadas no texto, alinhadas na margem esquerda, numeradas em algarismos arábicos, estes
alinhados na margem direita e entre parênteses. 

Alguns exemplos usando ambientes para inserir equações matemáticas:

\begin{align}
    & X_{t+1} = kx_t\left(1-x_t\right) \notag \\
    & E = mc^2\\
    & F = G \dfrac{m_1 m_2}{d^2}
\end{align}

\begin{gather}
    f(x)=\left\{\begin{aligned}
    & y_{ij} = 0\\
    & x = h + v
\end{aligned}\right.
\end{gather}

\begin{equation} 
    H = - \sum p(x) \log p(x)
\end{equation}

\begin{equation*}
    \nabla \times H = \dfrac{1}{c}\dfrac{\partial E}{\partial t},~~ \nabla\cdot H = 0
\end{equation*}

%----------------------------------------------------------------
\section{Ilustrações}
%----------------------------------------------------------------

Têm por objetivo exemplificar e/ou esclarecer o assunto que está
sendo abordado. Deve-se evitar o uso de ilustrações que não tenham sido objeto do trabalho.

\textbf{Exemplos:}

\begin{verbatim}
\begin{figure}[htb]{14cm}
\caption{Aqui entra o título da figura -- deve ser curto e objetivo}
\label{fig:fig1}
\fbox{\includegraphics[width=0.45\hsize]{logo_uerj_cinza.png}}%
\legend{Aqui entra a legenda da figura. Deve ser simples e sucinta
mas que garanta a compreensão da ilustração.
Qualquer explicação mais longa deve estar no corpo do texto.}
\source{Citação da fonte ou `O autor/A autora, ano'.}
\end{figure}
\end{verbatim}

O código acima produz a Figura~\ref{fig:fig1}.

\begin{figure}[htb]{14cm}
\caption{Aqui entra o título da figura -- deve ser curto e objetivo}\label{fig:fig1}
\fbox{\includegraphics[width=0.45\hsize]{config/UERJ/logo_uerj_cinza.png}}%
 \legend{Aqui entra a legenda da figura. Deve ser simples e sucinta mas que garanta a compreensão da ilustração.
Qualquer explicação mais longa deve estar no corpo do texto.}
\source{Citação da fonte ou `O autor/A autora, ano'.}
\end{figure}

Com o comando $\backslash$\texttt{subfloat[][]}\{\texttt{...}\} dentro do $\backslash$\texttt{begin}\{\texttt{figure}\}, é possível organizar mais de uma imagem no mesmo título. Cada ``subimagem'' será referenciada com uma letra a começar com \texttt{(a)}. Veja as figuras \ref{subrotulo1}, \ref{subrotulo2} e \ref{subrotulo3}.
\begin{verbatim}
\begin{figure}[htb]{16cm}
  \caption{Apresentações do logo da UERJ.} 
  \label{outro.rotulo}
  \subfloat[][]{\label{subrotulo1}%
    \fbox{\includegraphics[width=0.45\hsize]
            {logo_uerj_cinza.png}}}\hfill
  \subfloat[][]{\label{subrotulo2}%
    \fbox{\includegraphics[width=0.45\hsize]
            {marcadagua_uerj_cinza.png}}}\\
  \subfloat[][]{\label{subrotulo3}%
    \fbox{\includegraphics[width=0.45\hsize]
            {logo_uerj_cor.jpg}}}\hfill
  \legend{Logomarca da UERJ em diferentes 
    apresentações de cor: \subref{subrotulo1} 
    em tons de cinza (a partir da imagem colorida);
    \subref{subrotulo2} como marga d'água (a partir 
    da imagem preto e branco; \subref{subrotulo3} 
    colorido.}
  \source{\citeauthor{bib:logomarca}, 
          \citeyear{bib:logomarca}; 
          \citeauthor{bib:logomarca}, 
          \citeyear{bib:logomarca}.}
\end{figure}
\end{verbatim}

Não vale a pena brigar com o \LaTeX\ para posicionar a figura onde você quer. Normalmente, o \LaTeX\ colocará a figura na página seguinte, salvo exceções. Isto tem  a ver com a proporção de texto e figura presentes na página. Quando a figura ocupa uma área da página muito superior ao correspondente de texto, o \LaTeX\ deixa a figura isolada em uma nova página como, por exemplo, a Figura~\ref{fig:outro.rotulo}.

\begin{figure}[htb]{16cm}
  \caption{Apresentações do logo da UERJ.} 
  \label{fig:outro.rotulo}
  \subfloat[][]{\label{subrotulo1}%
    \fbox{\includegraphics[width=0.45\hsize]
            {config/UERJ/logo_uerj_cinza.png}}}\hfill
  \subfloat[][]{\label{subrotulo2}%
    \fbox{\includegraphics[width=0.45\hsize]
            {config/UERJ/marcadagua_uerj_cinza.png}}}\\
  \subfloat[][]{\label{subrotulo3}%
    \fbox{\includegraphics[width=0.45\hsize]
            {config/UERJ/logo_uerj_cor.jpg}}}\hfill
  \legend{Logomarca da UERJ em diferentes 
    apresentações de cor: \subref{subrotulo1} 
    em tons de cinza (a partir da imagem colorida);
    \subref{subrotulo2} como marga d'água (a partir 
    da imagem preto e branco; \subref{subrotulo3} 
    colorido.}
  \source{\citeauthor{bib:logomarca}, 
          \citeyear{bib:logomarca}; 
          \citeauthor{bib:logomarca}, 
          \citeyear{bib:logomarca}.}
\end{figure}




Texto do capítulo. Texto, texto Algoritmo \ref{alg:modelo}. Texto.

\begin{algorithm}[!ht]
    \caption{Título do algoritmo.} \label{alg:modelo}
    \begin{pseudocode}
      \LinhaEmBranco
      \Documentacao
        \Titulo{Nome do algoritmo\\}
        \Proposito{Propósito do algoritmo.\\}
        \Metodo{Método utilizado no algoritmo.\\}
        \Entradas{
          a, m: multiplicador e módulo\\
          n0: semente\\
          i: contador auxiliar\\
        }
        \Saidas{
          n: número aleatório\\
        }
        \Observacoes{Observações, restrições e requisitos.\\}
        \Algoritmo{Identificação}
          \Declarar{$a, m, i$}{numéricos}{}{}
          \Declarar{$n0, n$}{numéricos}{}{}
          \Ins{$m \leftarrow 13$}
          \Ins{$n0 \leftarrow 1$}
          \ParaDeAtePasso[para cada possível valor de `a']{$a$}{2}{$m-1$}{}
            \Escrever{``a = '', $a$, ``: n = \{''}
            \Ins[reinicia a geração com a semente n0]{$n \leftarrow n0$}
            \ParaDeAtePasso{$i$}{0}{$m-1$}{}
              \Ins[gerador de números aleatórios]{$n \leftarrow resto(a*n, m)$}
              \SeEntao[se fim da sequencia ...]{$n == n0$}
                \Escrever{$n$,``\}''}
                \Parar
              \Senao
                \Escrever{$n$}
              \FimSe
            \FimPara
          \FimPara
        \Continua
    \end{pseudocode}
\end{algorithm}

\alglinenumbersoff
\begin{algorithm*}[!ht]
    \caption{Título do algoritmo. (continuação)}
    \begin{pseudocode*}
        \LinhaEmBranco
        \Continuacao
%          \LinhaEmBranco
          \Ins{$a \leftarrow 1$}
          \Enquanto[comentário]{$a<10$}
            \Escrever{$a$}
            \Ins{$a \leftarrow a+1$}
          \FimEnquanto
          \Ins{$a \leftarrow 1$}
          \Repetir[comentário]
            \Escrever{$a$}
            \Ins{$a \leftarrow a+1$}
          \AteQue{$a\ge10$}
          \LinhaEmBranco
          \Ins{$a \leftarrow 1$}
          \Fazer[comentário]
            \Escrever{$a$}
            \Ins{$a \leftarrow a+1$}
          \Enquanto{$a<10$}
        \FimAlgoritmo
      \FimDocumentacao
    \end{pseudocode*}
\end{algorithm*}

\vfill~\\


%----------------------------------------------------------------
\section{Tabelas}
%----------------------------------------------------------------

Apresentam apenas informações estatísticas. As tabelas seguem as mesmas regras das ilustrações quanto a identificação, apresentação gráfica, localização e disposição.

\begin{table}[htb]{16cm}
    \caption{\protect Algumas disciplinas do Programa de Pós-Graduação em Ciências Computacionais.}
    \label{tab:casos_sarampo}
    \hfill\begin{tabular}{l|c}
        Disciplina & Grupo \\
        \hline 
         Álgebra Linear: Aspectos Teóricos e Computacionais & I\\
         Algoritmos & I\\
         Programação Paralela e Distribuída & II\\
         Probabilidade e Estatística & II\\
         Redes Neuronais & III\\
         Visão Computacional & III\\
         Estudo Dirigido I & IV
    \end{tabular}\hfill
    \legend{Algumas das disciplinas ofertadas pelo Programa de Pós-Graduação em Ciências Computacionais e o grupo ao qual pertencem.} 
    \source{O autor, 2021.}
\end{table}


%----------------------------------------------------------------
\section{Referências}
%----------------------------------------------------------------

No arquivo \textit{exemplos.bib} podem-se observar alguns exemplos de como formatar as referências a serem utilizadas no trabalho. 

Para obter mais detalhes, acessar o Roteiro para apresentação das teses e dissertações da Universidade do Estado do Rio de Janeiro através do link: \url{http://www.bdtd.uerj.br/roteiro_uerj_web.pdf}.

%================================================================
\chapter*{Conclusão} %não deve ser numerado
%================================================================

Texto da conclusão.
