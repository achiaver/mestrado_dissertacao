% ----------------------------------------------------------------
% ELEMENTOS POS-TEXTUAIS
% ----------------------------------------------------------------

\backmatter % inicia a área dos elementos pós-textuais


%===========================================================
% Referencias via BibTeX
%===========================================================

\citeoption{abnt-options4}
\bibliography{bibs/abnt-options4,
              bibs/bibliografia}
% \bibliographystyle{abntex2-num}
% \bibliography{bibliografia}

%===========================================================
\postextualchapter*{Glossário} % elemento opcional
%===========================================================

\definicao{termo 1}{significado}
\definicao{termo 2}{significado}
\definicao{termo 3}{significado}

% ----------------------------------------------------------------
% Apêndices (opcionais)
% ----------------------------------------------------------------

\appendix % inicia os apêndices

%===========================================================
\postextualchapter{Primeiro apêndice}
%===========================================================

Os apêndices são elementos opcionais, complementares, de caráter informativo, elaborados pelo próprio autor, como, por exemplo, questionários, formulários, textos etc. Sua exclusão não prejudica o conteúdo do trabalho. Caso haja somente um apêndice, este não deve ser numerado. Para isso, é necessário comentar a linha $565$ do arquivo repUERJ.cls e retirar o comentário da linha $566$ do mesmo arquivo.

% ----------------------------------------------------------
\section{Primeira seção}
% ----------------------------------------------------------

Texto da primeira seção.

% ----------------------------------------------------------
\subsection{Primeira subseção}
% ----------------------------------------------------------

Texto da primeira subseção.

% ----------------------------------------------------------
\subsubsection{Primeira subsubseção}
% ----------------------------------------------------------

Texto da primeira subsubseção.


%===========================================================
\postextualchapter{Segundo apêndice}
%===========================================================

% ----------------------------------------------------------
\section{Primeira seção}
% ----------------------------------------------------------

Texto da primeira seção.

% ----------------------------------------------------------
\subsection{Primeira subseção}
% ----------------------------------------------------------

Texto da primeira subseção.

% ----------------------------------------------------------
\subsubsection{Primeira subsubseção}
% ----------------------------------------------------------

Texto da primeira subsubseção.

% ----------------------------------------------------------------
% Anexos (opcionais)
% ----------------------------------------------------------------

\annex % inicia os anexos

%===========================================================
\postextualchapter{Primeiro anexo}
%===========================================================

Os anexos são elementos opcionais, complementares, de caráter ilustrativo e/ou comprobatório do texto. O anexo difere do apêndice por não ser elaborado pelo autor da tese ou dissertação. Caso haja somente um anexo, este não deve ser numerado, como ocorre no apêndice.
Caso haja mais de um anexo, é necessário comentar a linha $575$ do arquivo repUERJ.cls e retirar o comentário da linha $574$ do mesmo arquivo.

% ----------------------------------------------------------
\section{Primeira seção}
% ----------------------------------------------------------

Texto da primeira seção.

% ----------------------------------------------------------
\subsection{Primeira subseção}
% ----------------------------------------------------------

Texto da primeira subseção.

% ----------------------------------------------------------
\subsubsection{Primeira subsubseção}
% ----------------------------------------------------------

Texto da primeira subsubseção.



% %===========================================================
% \postextualchapter{Segundo anexo}
% %===========================================================

% % ----------------------------------------------------------
% \section{Primeira seção}
% % ----------------------------------------------------------

% Texto da primeira seção.

% % ----------------------------------------------------------
% \subsection{Primeira subseção}
% % ----------------------------------------------------------

% Texto da primeira subseção.

% % ----------------------------------------------------------
% \subsubsection{Primeira subsubseção}
% % ----------------------------------------------------------

% Texto da primeira subsubseção.


