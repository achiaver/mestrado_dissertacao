% ----------------------------------------------------------
% ELEMENTOS PRE-TEXTUAIS
% ----------------------------------------------------------

\frontmatter % inicia a área dos elementos pré-textuais

% O tipo de documento (tese ou dissertação) deve ser definido após a linha 165

% ----------------------------------------------------------
% Capa e a folha de rosto
% ----------------------------------------------------------

\capa
\folhaderosto

% ----------------------------------------------------------
% Inserir a ficha catalográfica
% ----------------------------------------------------------

% A biblioteca deverá providenciar a ficha catalográfica. 
% Salve a ficha no formato PDF. Use o nome do arquivo PDF 
% como argumento do comando. 
% Exemplo: ficha catalográfica no arquivo 'ficha.pdf'
%     \fichacatalografica{ficha.pdf}
%
% Enquanto não possuir a ficha catalográfica, use o comando sem
% argumentos.
% ---

\fichacatalografica{}


% ----------------------------------------------------------
% Folha de aprovação
% ----------------------------------------------------------

\begin{folhadeaprovacao}
  \assinatura{Cargo Título Nome Completo}
             {Unidade -- Instituição}
  \assinatura{Cargo Título Nome Completo}
             {Unidade -- Instituição}
  \assinatura{Cargo Título Nome Completo}
             {Unidade -- Instituição}
  \assinatura{Cargo Título Nome Completo}
             {Unidade -- Instituição}
\end{folhadeaprovacao}


% ----------------------------------------------------------
% Dedicatória
% ----------------------------------------------------------

\pretextualchapter{Dedicatória}
\vfill
Texto da dedicatória (opcional).


% ----------------------------------------------------------
% Agradecimentos
% ----------------------------------------------------------

\pretextualchapter{Agradecimentos}

Texto de agradecimento (opcional).


% ----------------------------------------------------------
% Epigrafe (opcional)
% ----------------------------------------------------------

\pretextualchapter{}
  \vfill
  \begin{flushright}
 Pensamento, reflexão (opcional). \\  %citação sem aspas   
    \textit{autor}
  \end{flushright}


% ----------------------------------------------------------
% RESUMO
% ----------------------------------------------------------

\pretextualchapter{Resumo}
\referencia % linha em branco depois


Texto do resumo em português.


\imprimirchaves % linha em branco antes

% ----------------------------------------------------------
% Abstract
% ----------------------------------------------------------

\pretextualchapter{Abstract}
\reference % linha em branco depois


Texto do resumo em inglês.


\printkeys % linha em branco antes

% ----------------------------------------------------------
% Listas de ilustrações e tabelas
% ----------------------------------------------------------

\listadefiguras    %\begin{figure}{largura}...\end{figure}
\listadetabelas    %\begin{table}{largura}...\end{table}
% \listadegraficos   %\begin{graph}{largura}...\end{graph}

% ----------------------------------------------------------
% Outras listas
% ----------------------------------------------------------

\listadealgoritmos % opcional %\begin{algorithm}...\end{algorithm}

% ----------------------------------------------------------
% Lista de abreviaturas e siglas (opcional)
% ----------------------------------------------------------

\pretextualchapter{Lista de abreviaturas e siglas}
    \abreviatura{sigla 1}{por extenso}
    \abreviatura{sigla 2}{por extenso}
    \abreviatura{sigla 3}{por extenso}

% ----------------------------------------------------------
% Lista de simbolos (opcional)
% ----------------------------------------------------------

\pretextualchapter{Lista de símbolos}
    \simbolo{$simbolo 1$}{significado e/ou valor}
    \simbolo{$simbolo 2$}{significado e/ou valor}
    \simbolo{$simbolo 3$}{significado e/ou valor}

% ----------------------------------------------------------
% Sumario
% ----------------------------------------------------------

\sumario
