% ---------------------------------------------------------------
% ---------------------------------------------------------------
% Packages content
% ---------------------------------------------------------------
% ---------------------------------------------------------------

\usepackage[main=brazil,english]{babel}  % adequação para o português Brasil
\usepackage[utf8]{inputenc} % Determina a codificação utilizada
                            % (conversão automática dos acentos)
\usepackage{makeidx}        % Cria o índice
\usepackage{hyperref}       % Controla a formação do índice
\usepackage{indentfirst}    % Indenta o primeiro paragrafo de
                            % cada seção.
\usepackage{graphicx}       % Inclusão de gráficos
%\graphicspath{{images/}}

\usepackage{subfig}
\usepackage{multirow}
\usepackage{amsmath,amssymb}  % pacotes matemáticos
\usepackage[alf]{abntex2cite} % pacote para citacoes autor-data
% \usepackage[num]{abntex2cite} % pacote para citacoes com numeracao
% \citebrackets[]
\usepackage{tikz}				    % Awesome drawings
\usetikzlibrary{calc}               % to get node width and height
\usetikzlibrary{positioning}        % to use the relative positioning commands
\usetikzlibrary{arrows.meta}
\usetikzlibrary{mindmap}
\usetikzlibrary{shapes.geometric}   % to use diamond shape
\usetikzlibrary{backgrounds}        % for frames
\usetikzlibrary{shadows}            % for shadows
\usetikzlibrary{arrows}
\usetikzlibrary{decorations.markings}
\usepackage{tikz-3dplot}

\pdfstringdefDisableCommands{\let\uppercase\relax}
\pdfstringdefDisableCommands{\let\uline\relax}

%\usepackage[table]{xcolor}

\usepackage[font=default,frame=no]{config/UERJ/repUERJformat} % pacote para as 
                                                  % normas da UERJ
% ---
% pacote auxiliar para elaboração de pseudocódigos
% este pacote pode ser retirado caso nao se planeje
% elaborar pseudocódigos
% ---
\usepackage[dots=yes]{config/UERJ/repUERJpseudocode}

\usepackage[maxfloats=25]{morefloats}
\usepackage{array}
\setlength\extrarowheight{2pt}
% \usepackage[subrefformat=parens, labelformat=parens]{subfig}
% \captionsetup[subfigure]{justification=centering}
% \newcommand{\subfigureautorefname}{\figureautorefname}
% \newcommand{\lstnumberautorefname}{\figureautorefname}



% ----------------------------------------------------------------
% Este trecho de comandos pode ser apagado
% ----------------------------------------------------------------
\usepackage[brazilian,hyperpageref]{backref}
% Configurações do pacote backref
% Usado sem a opção hyperpageref de backref
\renewcommand{\backrefpagesname}{Citado na(s) página(s):~}
% Texto padrão antes do número das páginas
\renewcommand{\backref}{}
% Define os textos da citação
\renewcommand*{\backrefalt}[4]{
\ifcase #1 %
Nenhuma citação no texto.%
\or
Citado na página #2.%
\else
Citado #1 vezes nas páginas #2.%
\fi}%